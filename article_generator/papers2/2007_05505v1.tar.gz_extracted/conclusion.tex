\section{Conclusion}
Incident management is a key part of building and operating large-scale cloud services. Prior work on incident management has focused on specific aspects such as triaging, predicting incidents and root causing. In this paper, we propose \softner{}, a deep learning based unsupervised framework for knowledge extraction from incidents. \softner{} uses structural patterns and label propagation for bootstrapping the training data. It then incorporates a novel multi-task BiLSTM-CRF model for automated extraction of named-entities from the incident descriptions. We have evaluated \softner{} on the incident data from \CompanyX{}, a major cloud service provider. Our evaluation shows that even though \softner{} is fully unsupervised, it has a high precision of 0.96 (at rank 50) for learning entity types from the unstructured incident data. Further, our multi-task model architecture outperforms existing state of the art models in entity extraction. It is able to achieve an average F1 score of 0.957 and a weighted average F1 score of 0.968. As shown by the manual evaluation, \softner{} is also able to generalize beyond the structural patterns which were used to bootstrap. We have deployed \softner{} at \CompanyX{}, where it has been used for knowledge extraction from over 400 incidents. We also discuss several real-world applications of knowledge extraction. Lastly, we show that the extracted knowledge can be used for building significantly more accurate models for critical incident management tasks like triaging.