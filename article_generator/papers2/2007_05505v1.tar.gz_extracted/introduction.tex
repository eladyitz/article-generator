\section{Introduction}
In the last decade, two major paradigm shifts have revolutionized the software industry. First is the move from boxed software products to \emph{services}. Large software organizations like Adobe\iffalse\footnote{https://producthabits.com/adobe-95-billion-saas-company/}\fi, Autodesk\footnote{https://www.barrons.com/articles/autodesks-bet-on-the-cloud-will-generate-big-returns-for-shareholders-1443248123} and Microsoft\footnote{https://www.pcworld.com/article/2038194/microsoft-says-its-boxed-software-probably-will-be-gone-within-a-decade.html} which pre-date the internet revolution have been aggressively trying to move from selling boxed products to subscription based services. This has primarily been driven by the benefits of subscription based services. In terms of technical advantages, these services can always be kept up-to-date, they do not need to wait for the next shipping cycle to install security patches, and, lastly, the telemetry from these services provide invaluable insights about the usage of the products. On the monetary side, these subscription based services provide recurring revenue, instead of a one time sale. Also, it allows companies to reach their customers directly, bypassing the vendors and retailers. The second major shift has been the widespread adoption of \emph{public clouds}. More and more software companies are moving from on-premises data centers to public clouds like Amazon AWS, Google Cloud, Microsoft Azure, etc. Gartner has forecasted\footnote{https://www.gartner.com/en/newsroom/press-releases/2019-11-13-gartner-forecasts-worldwide-public-cloud-revenue-to-grow-17-percent-in-2020} the public cloud market to grow to about \$$266$ billion revenue in $2020$, out of which about 43\% revenue will be from the Software as a Service (SaaS) segment. The cloud revolution has enabled companies like Netflix, Uber, etc. to build internet scale products without having to stand up their own infrastructure.

While the paradigm shifts have revolutionized the software industry, they have also had a transformational effect on the way software is developed, deployed and maintained. For instance, software engineers no longer develop monolithic software. They build services which have dependencies on several 1st part and 3rd party services and APIs. Typically, any web application will leverage cloud services for basic building blocks like storage (relational and blob), compute, authentication. These complex dependencies introduce a bottleneck where a single failure can have a cascading effect. In 2017, a small typo led to a major outage in the AWS S3 storage service, which ended up costing over \$$150$ million to customers like Slack, Medium, etc\footnote{https://www.wsj.com/articles/amazon-finds-the-cause-of-its-aws-outage-a-typo-1488490506}. Similarly, Google Cloud, had a config error affect Google cloud for 4+ hours \footnote{https://www.zdnet.com/article/google-details-catastrophic-cloud-outage-events-promises-to-do-better-next-time/}. Microsoft had a glitch in their Active Directory last October\footnote{https://www.theregister.co.uk/2019/10/18/microsoft\_azure\_mfa/}, which locked out customers from accessing their Office 365 and Azure accounts. These outages are inevitable and can be caused by various factors such as code bugs, mis-configurations \cite{mehta2020rex} or even environmental factors.

To keep up with these changes, DevOps processes and platforms have also evolved over time \cite{lipredicting, dang2019aiops, kumar2019building}. Most software companies these days have incident management and on-call duty as a part of the DevOps workflow. Feature teams have on-call rotations where engineers building the features take on incident management responsibilities for their products. Depending on the severity of the incident and the service level objective (SLO), they would need to be available 24x7 and respond to incidents in near-real time. The key idea behind incident management is to reduce impact on customers by mitigating the issue as soon as possible. We discuss the incident life-cycle and some of the associated challenges in detail in Section~2. Prior work on incident management has largely focused on two challenges: incident triaging \cite{ContinuousTriageASE2019, EmpiricalIcMICSE2019} and diagnosis \cite{nair2015learning, bansal2019decaf, luo2014correlating}. Chen et al. \cite{ContinuousTriageASE2019} did an empirical study where they found that upto 60\% of the incidents can be mis-triaged. They proposed DeepCT, a deep learning approach for automated incident triaging using incident data (title, summary, comments) and the environmental factors. 

In this work, we address the complimentary problem of \textbf{extracting structured knowledge from service incidents}. Based on our experience from operating web-scale cloud services at \CompanyX{}, incidents created by internal and external customers contain huge amount of unstructured information. Ideally, any knowledge extraction framework should have the following qualities:
\begin{enumerate}
    \item It should be \textbf{unsupervised} because it's laborious and expensive to annotate a large amount of training data. This is important since every service and team would have it's own vocabulary. For instance, the information or entities contained in incidents for the storage service is very different from the incidents for the compute service.
    \item It should be \textbf{domain agnostic} so that it can scale to a high volume and a large number of entity types. In the web domain, there are a small set of key entities such as people, places, organizations. However, in the incidents, we don't know these entities apriori.
    \item It should be \textbf{extensible}, we should be able to adapt the bootstrapping techniques to incidents from other services or even other data-sets such as bug reports. This is critical because each service could have it's unique vocabulary and terminology.
\end{enumerate}

We have designed \textbf{Soft}ware artifact \textbf{N}amed-\textbf{E}ntity \textbf{R}ecognition (\softner{}), a framework for unsupervised knowledge extraction from service incidents, which has these three qualities: unsupervied, domain agnostic, and extensible. We frame the knowledge extraction problem as a \textit{Named-entity recognition} task which has been well explored in the Information Retrieval domain \cite{nadeau2007survey, lample2016neural}. It leverages structural pattern extractors for bootstrapping the training data for knowledge extraction. Further, \softner{} incorporates a novel multi-task BiLSTM deep learning model with attention mechanism. We have evaluated and deployed SoftNER at \CompanyX{}, a major cloud service provider. We show that our unsupervised entity extraction has high precision. Our multi-task deep learning model also outperforms existing state of the art models on the NER task. Lastly, using the extracted knowledge, we are able to build more accurate models for key downstream tasks like incident triaging. In this work, we make the following main contributions:
\begin{enumerate}
    \item We propose \softner{}, the first approach for completely unsupervised knowledge extraction from service incidents.
    \item We build a novel multi-task learning based deep learning model which leverages not just the semantic features but also the data-types. It outperforms the existing state of the art NER models. 
    \item We do an extensive evaluation of \softner{} on over 8000 cloud service incidents from \CompanyX{}.
    \item Lastly, we have deployed \softner{} in production at \CompanyX{} where it has been used for knowledge extraction from over 400 incidents.
\end{enumerate}

The rest of the paper is organized as follows: In Section 2, we discuss insights from the incident management processes and the challenges at \CompanyX{}. Section 3 and 4 provide an overview, and details of our approach for knowledge extraction from incidents with \softner{}, respectively. In Section 5, we discuss the experimental evaluation of our approach. Section 6 describes the various applications and Section 7 discusses the related work. We conclude the paper in Section 8.