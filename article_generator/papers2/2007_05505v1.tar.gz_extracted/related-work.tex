\section{Related Work}
Incident management has recently become an active area of research in the software engineering community. Significant work has been done on specific aspects of incident management, such as automated triaging of incidents, incident diagnosis and detection. Our work is complementary to the existing work on incident management since we focus on the fundamental problem of structured knowledge extraction from incidents. We also show that the knowledge extracted by \softner{} can be used to build more accurate models for these incident management tasks. \softner{} is inspired by the work on knowledge extraction in the Information Retrieval community. There have been decades of research on knowledge extraction from web data. However, majority of the research in the web space has been focused on supervised or semi-supervised entity extraction, limited to a known set of entities such as people, organizations and places. Also, unlike web pages, in software artifacts like incidents, the vocabulary is not limited to English and other human languages. Incidents contain not just textual information but also other entities such as GUIDs, Exceptions, IP Addresses, etc. Hence, \softner{} leverages a novel data-type aware deep learning model for knowledge extraction. Next, we discuss prior work in different domains which is related to this work.

\textbf{Incident management}: Recent work on incident management has been focused on the problem of triaging incidents to the correct teams. As per the empirical study by Chen et al. \cite{EmpiricalIcMICSE2019}, miss-triaging of incidents happen quite frequently and can lead up to a delay of over 10X in triaging time, apart from the lost revenue and customer impact. To solve this problem, they have also proposed DeepCT \cite{ContinuousTriageASE2019}, a deep learning based method for routing of incidents to the correct teams. They evaluate the model on several large-scale services and are able to achieve a high mean accuracy of up to 0.7. There has also been significant amount of work done on diagnosing and root causing of service incidents. Nair et al. \cite{nair2015learning} uses hierarchical detectors based on time-series anomaly detection for diagnosing incidents in services. DeCaf \cite{bansal2019decaf} uses random forest models for automatically detecting performance issues in large-scale cloud services. It also categorizes the detected issues into the buckets of new, known, regressed, resolved issues based on trend analysis. Systems such as AirAlert \cite{chen2019outage} have been built for predicting critical service incidents, called outages, in large scale services. It correlates signals and dependencies from across the cloud system and uses machine learning for predicting outages. Different from existing work, we focus on the fundamental problem of knowledge extraction from the service incidents. As we show in Section 5.4, using the named-entities extracted by \softner{}, we can build significantly more accurate models for these incident management tasks.

\textbf{Bug reports}: Significant amount of research has been done on bug reports in the traditional software context. \softner{} is inspired by InfoZilla \cite{bettenburg2008extracting} which leverages heuristics and regular expressions for extracting four elements from Eclipse bug reports: patches, stack traces, code and lists. Unlike InfoZilla, we build a completely unsupervised deep learning based framework for extracting entities from incidents. This enables \softner{} to extract hundreds of entities without requiring any prior knowledge about them. Our work also targets incidents, which are more complex than bugs because of numerous layers of dependencies and, also, the real-time mitigation requirements. Bug triage has been an active area of research \cite{anvik2006should, tian2016learning, bortis2013porchlight, wang2014fixercache} in the software engineering community. Other aspects of bug reports such as bug classification \cite{zhou2016combining} and bug fix prediction time \cite{ardimento2017knowledge} have also been explored. Similar to incidents, existing work on bug reports have largely used the unstructured attributes like bug description as it is. Even though in this work, we have focused on incidents, \softner{} can be applied to bug reports for extracting structured information and building models for tasks like triaging, classification, etc.

\textbf{Information retrieval}: Knowledge and entity extraction has been studied in depth by the information retrieval community. Search engines like Google and Bing rely heavily on entity knowledge bases for tasks like intent understanding \cite{pantel2012mining} and query reformulation \cite{xu2008entity}. Several supervised and semi-supervised methods have been proposed for entity extraction from the web corpora \cite{florian2003named, mccallum2003early, pacsca2007weakly, vyas2009semi}. Supervised methods require large amount of training data which can be cost prohibitive to collect. Hence, the search engines commonly use semi-supervised methods which leverage a small seed set to bootstrap the entity extraction process. For instance, the expert editors would seed the entity list for a particular entity type, let's say fruits with some initial values such as \{apple, mango, orange\}. Then using pattern or distributional extractors, the list would be expanded to cover the entire list of fruits. In this work, our goal was to build a fully unsupervised system where we don't need any pre-existing list of entity types or seed values. This is primarily because every service and organization is unique and manually bootstrapping \softner{} would be very laborious. Also, unlike web pages, incident reports are not composed of well-formed and grammatically correct sentences. Hence, we propose a novel multi-task based deep learning model which uses both the semantic context and the data-type of the tokens for entity extraction.