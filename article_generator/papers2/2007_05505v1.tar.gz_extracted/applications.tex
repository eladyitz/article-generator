\section{Applications}
Automated extraction of structured knowledge from the service incidents unlocks several applications and scenarios:

\textbf{Incident summarization} - As shown in Table \ref{entity-examples}, \softner{} is able to extract key information from the incidents. This information can be added to the incidents as a summary for a quick overview for the on-call engineers. Otherwise, parsing the verbose incident descriptions to understand the issue and to locate the key information could be overwhelming. We have already enabled this scenario for one cloud service at \CompanyX{} and the feedback has been positive. We have added the summary for over 400 incidents.

\textbf{Automated health-checks} - Using \softner{}, we are able to extract key information such as \textit{Resource Id}, \textit{IP Addresses}, \textit{Subscription Id}, etc. This enables us to automate the manual tasks done by the on-call engineers for investigating the incidents. For instance, they will often pull up telemetry and logs for the affected resources. Or, they might look up the resource allocation for a given subscription. With the named entities, we can run these health checks automatically before the on-call engineers are engaged.

\textbf{Bug reports} - Even though this work is motivated by the various problems associated with incident management, the challenges with lack of structure applies to bug reports as well. We plan to evaluate \softner{} on bug reports at \CompanyX{} and, also, on the publicly available bug report datasets.

\textbf{Knowledge graph} - Currently, given an incident, \softner{} extracts a flattened list of entity name and value pairs. As a next step, we plan to work on relation extraction to better understand which entities are related. For instance, it will be quite useful to be able to map a given Identifier and IP Address to the same Virtual Machine resource. This would also allow us to incorporate the entities into a knowledge graph. This can unlock various applications such as entity disambiguation and linking.

\textbf{Better tooling} - The knowledge extracted by \softner{} can also be used to improve the existing incident reporting and management tools. We are able to build a named-entity set at any granularity (service, organization, feature team) in a fully unsupervised manner. These entities can then be incorporated into the incident report form as well, where some of these entities can even be made mandatory. We have already started working on this scenario with the feature teams which own incident reporting tools at \CompanyX{}.

\textbf{Type-aware models} - The mutli-task deep learning model architecture used by \softner{} uses both the semantic and the data-type context for entity extraction. As per our knowledge, this is the first usage of a multi-task and type-aware architecture in the software engineering domain. Given that software code and programs are typed, this model can potentially be used in other applications like code summarization where we can have a generative task along with a similar classification task for data-type prediction of the code tokens.

\textbf{Predictive tasks} - In this work, we have shown that the knowledge extracted by \softner{} can be used to build more accurate machine learning models for incident triaging. Similarly, we can build models to automate other tasks such as severity prediction, root causing, abstractive summarization \cite{paulus2017deep}, etc.